\chapter{Linear equations and matrices}

\section{Linear equations}

Let $m$ and $n$ be positive integers. A \keyword{linear equation in $n$ unknowns} is an equation on the form
%
\begin{equation*}
    l \colon a_1 x_1 + \cdots + a_n x_n = b,
\end{equation*}
%
where $a_1, \ldots, a_n, b \in \fieldF$. A \keyword{solution} to $l$ is an element $v = (v_1, \ldots, v_n) \in \fieldF^n$ such that
%
\begin{equation*}
    a_1 v_1 + \cdots + a_n v_n = b.
\end{equation*}
%
A \keyword{system of linear equations in $n$ unknowns} is a tuple $L = (l_1, \ldots, l_m)$, where each $l_i$ is a linear equation in $n$ unknowns. An element $v \in \fieldF^n$ is a \keyword{solution} to $L$ if it is a solution to each linear equation $l_1, \ldots, l_m$.

Let $L$ and $L'$ be systems of linear equations in $n$ unknowns. We say that $L$ and $L'$ are \keyword{solution equivalent} if they have the same solutions. Furthermore, we say that they are \keyword{combination equivalent} if each equation in $L'$ is a linear combination of the equations in $L$, and vice versa. Clearly, if $L$ and $L'$ are combination equivalent they are also solution equivalent, but the converse does not hold.


\section{Matrices}

For $m,n \in \naturals$ we denote by $\mat{m,n}{R}$ the set of $m \times n$ matrices over $R$. In the case where $R = \fieldF$, it is well-known that a system of linear equations is equivalent to a matrix equation on the form $Ax = b$, where $A \in \mat{m,n}{\fieldF}$, $x \in \fieldF^n$ and $b \in \fieldF^m$. Recall the \keyword{elementary row operations} on $A$:
%
\begin{enumerate}
    \item multiplication of one row of $A$ by a nonzero scalar,
    \item addition to one row of $A$ a scalar multiple of another (different) row, and
    \item interchange of two rows of $A$.
\end{enumerate}
%
If $e$ is an elementary row operation, we write $e(A)$ for the matrix obtained when applying $e$ to $A$. Clearly each elementary row operation $e$ has an \enquote{inverse}, i.e. an elementary row operation $e'$ such that $e'(e(A)) = e(e'(A)) = A$. Two matrices $A,B \in \mat{m,n}{\fieldF}$ are called \keyword{row-equivalent} if $A$ is obtained by applying a finite sequence of elementary row operations to $B$ (and vice versa, though this need not be assumed since each elementary row operation has an inverse).

Clearly, if $A, B \in \mat{m,n}{\fieldF}$ are row-equivalent, then the systems of equations $Ax = 0$ and $Bx = 0$ are combination equivalent, hence have the same solutions.

An \keyword{elementary matrix} is a matrix obtained by applying a single elementary row operation to the identity matrix $I$. It is easy to show that if $e$ is an elementary row operation and $E = e(I) \in \mat{m}{\fieldF}$, then $e(A) = EA$ for $A \in \mat{m,n}{\fieldF}$. If also $B \in \mat{m,n}{\fieldF}$, then $A$ and $B$ are row-equivalent if and only if $A = PB$, where $P \in \mat{m}{\fieldF}$ is a product of elementary matrices.

We now show that every matrix is row-equivalent to a matrix with a particularly simple form:

\begin{definition}\label{def:mydef}
    A matrix $H \in \mat{m,n}{\fieldF}$ is called \keyword{row-reduced} if
    %
    \begin{enumdefinition}
        \item \label{enumdef:myenum} the first nonzero entry of each nonzero row in $H$ is $1$, and
        \item each column of $H$ containing the leading nonzero entry of some row has all its other entries equal $0$.
    \end{enumdefinition}
    %
    If $H$ is row-reduced, it is called a \keyword{row-reduced echelon matrix} if it also has the following properties:
    %
    \begin{enumdefinition}[resume]
        \item Every row of $H$ only containing zeroes occur below every row which has a nonzero entry, and
        \item if rows $1, \ldots, r$ are the nonzero rows of $H$, and if the leading nonzero entry of row $i$ occurs in column $k_i$, then $k_1 < \cdots < k_r$.
    \end{enumdefinition}
\end{definition}

\begin{proposition}
    Every matrix in $\mat{m,n}{\fieldF}$ is row-equivalent to a unique row-reduced echelon matrix.
\end{proposition}

\begin{proof}
    The usual Gauss--Jordan elimination algorithm proves existence. If $H, K \in \mat{m,n}{R}$ are row-equivalent row-reduced echelon matrices, we claim that $H = K$. We prove this by induction in $n$. If $n = 1$ then this is obvious, so assume that $n > 1$. Let $H_1$ and $K_1$ be the matrices obtained by deleting the $n$th column in $H$ and $K$ respectively. Then $H_1$ and $K_1$ are also row-equivalent\footnote{It should be obvious that deleting columns preserves row-equivalence, but we give a more precise argument: If $P \in \mat{m}{\fieldF}$ is a product of elementary matrices and $a_1, \ldots, a_n \in \fieldF^m$ are the columns in $A$, then the columns in $PA$ are $Pa_1, \ldots, Pa_m$. Thus elementary row operations are applied to each column independently of the other columns.} and row-reduced echelon matrices, so by induction $H_1 = K_1$. Thus if $H$ and $K$ differ, they must differ in the $n$th column.

    Let $H_2$ be the matrix obtained by deleting columns in $H$, only keeping those columns containing pivots, as well as keeping the $n$th column. Define $K_2$ similarly. Thus we have deleted the same columns in $H$ and $K$, so $H_2$ and $K_2$ are also row-equivalent. Say that the number of columns in $H_2$ and $K_2$ is $r+1$, and write the matrices on the form
    %
    \begin{equation*}
        H_2
            = \begin{pmatrix}
                I_r & h \\
                0   & h'
            \end{pmatrix}
        \quad \text{and} \quad
        K_2
            = \begin{pmatrix}
                I_r & k \\
                0   & k'
            \end{pmatrix},
    \end{equation*}
    %
    where $h,k \in \fieldF^r$ and $h',k' \in \fieldF^{m-r}$ are column vectors. Since $H_2$ and $K_2$ are row-equivalent, the systems $H_2 x = 0$ and $K_2 x = 0$ are solution equivalent. If $h' = 0$, then $H_2 x = 0$ has the solution $(-h,1)$. But this is also a solution to $K_2 x = 0$, so $h = k$ and $k' = 0$. If $h' \neq 0$, then $H_2 x = 0$ only has the trivial solution. But then $K_2 x = 0$ also only has the trivial solution, and hence $k' \neq 0$. But that must be because both $H_2$ and $K_2$ has a pivot in the rightmost column, so also in this case $H_2 = K_2$.
\end{proof}

Next we study when square matrices are invertible. The main result says for a square matrix to be invertible, it suffices that it has either a left- or a right-inverse. Since (as we will see in {par:matrix-rep} TODO) square matrices are precisely the linear endomorphisms on finite-dimensional vector spaces, this shows that for such an endomorphism to be invertible, it suffices that it has either a left- or a right-inverse. This result is also a direct consequence of the rank--nullity theorem, see e.g. \textcite[Corollary~2.9]{romanlinalg} (TODO and local ref).

Notice that elementary matrices are invertible since elementary row operations are invertible.

\begin{lemma}
    If $A \in \mat{n}{\fieldF}$, then the following are equivalent:
    %
    \begin{enumlemma}
        \item \label{enum:lemma-A-invertible} $A$ is invertible,
        \item \label{enum:lemma-A-equivalent-to-I} $A$ is row-equivalent to $I_n$,
        \item \label{enum:lemma-A-elementary-matrix-product} $A$ is a product of elementary matrices, and
        \item \label{enum:lemma-only-trivial-solution} the system $Ax = 0$ has only the trivial solution $x = 0$.
    \end{enumlemma}
\end{lemma}

\begin{proof}
\begin{proofsec*}
    \item[\itemref{enum:lemma-A-invertible} $\iff$ \itemref{enum:lemma-A-equivalent-to-I}]
    Let $H \in \mat{n}{\fieldF}$ be a row-reduced echelon matrix that is row-equivalent to $A$. Then $H = PA$, where $P \in \mat{n}{\fieldF}$ is a product of elementary matrices. Then $A = \inv{P} H$, so $A$ is invertible if and only if $H$ is. But the only invertible row-reduced echelon matrix is the identity matrix, so \itemref{enum:lemma-A-invertible} and \itemref{enum:lemma-A-equivalent-to-I} are equivalent.
    
    \item[\itemref{enum:lemma-A-equivalent-to-I} $\implies$ \itemref{enum:lemma-A-elementary-matrix-product}]
    As above, there exists a product $P$ of elementary matrices such that $I_n = PA$, so $A = \inv{P}$.

    \item[\itemref{enum:lemma-A-elementary-matrix-product} $\implies$ \itemref{enum:lemma-A-invertible}]
    This is obvious since elementary matrices are invertible.

    \item[\itemref{enum:lemma-A-equivalent-to-I} $\iff$ \itemref{enum:lemma-only-trivial-solution}]
    If $A$ and $I_n$ are row-equivalent, then the systems $Ax = 0$ and $I_n x = 0$ have the same solutions. Conversely, assume that $Ax = 0$ only has the trivial solution. If $H \in \mat{m,n}{\fieldF}$ is a row-reduced echelon matrix that is row-equivalent to $A$, then $Hx = 0$ has no nontrivial solution. Thus if $r$ is the number of nonzero rows in $H$, then $r \geq n$. But then $r = n$, so $H$ must be the identity matrix.
\end{proofsec*}
\end{proof}


\begin{proposition}
    Let $A \in \mat{n}{\fieldF}$. Then the following are equivalent:
    %
    \begin{enumproposition}
        \item $A$ is invertible,
        \item $A$ has a left inverse, and
        \item $A$ has a right inverse.
    \end{enumproposition}
\end{proposition}

\begin{proof}
    If $A$ has a left inverse, then $Ax = 0$ has no nontrivial solution, so $A$ is invertible. If $A$ has a right inverse $B \in \mat{n}{\fieldF}$, i.e. $AB = I$, then $B$ has a left inverse and is thus invertible. But then $A$ is the inverse of $B$ and hence is itself invertible.
\end{proof}

\chapter{Linear equations and matrices}

\section{Linear equations}

Let $m$ and $n$ be positive integers. A \keyword{linear equation in $n$ unknowns} is an equation on the form
%
\begin{equation*}
    l \colon a_1 x_1 + \cdots + a_n x_n = b,
\end{equation*}
%
where $a_1, \ldots, a_n, b \in \fieldF$. A \keyword{solution} to $l$ is an element $v = (v_1, \ldots, v_n) \in \fieldF^n$ such that
%
\begin{equation*}
    a_1 v_1 + \cdots + a_n v_n = b.
\end{equation*}
%
A \keyword{system of linear equations in $n$ unknowns} is a tuple $L = (l_1, \ldots, l_m)$, where each $l_i$ is a linear equation in $n$ unknowns. An element $v \in \fieldF^n$ is a \keyword{solution} to $L$ if it is a solution to each linear equation $l_1, \ldots, l_m$.

Let $L$ and $L'$ be systems of linear equations in $n$ unknowns. We say that $L$ and $L'$ are \keyword{solution equivalent} if they have the same solutions. Furthermore, we say that they are \keyword{combination equivalent} if each equation in $L'$ is a linear combination of the equations in $L$, and vice versa. Clearly, if $L$ and $L'$ are combination equivalent they are also solution equivalent, but the converse does not hold.


\section{Matrices}

For $m,n \in \naturals$ we denote by $\mat{m,n}{R}$ the set of $m \times n$ matrices over $R$. In the case where $R = \fieldF$, it is well-known that a system of linear equations is equivalent to a matrix equation on the form $Ax = b$, where $A \in \mat{m,n}{\fieldF}$, $x \in \fieldF^n$ and $b \in \fieldF^m$. Recall the \keyword{elementary row operations} on $A$:
%
\begin{enumerate}
    \item multiplication of one row of $A$ by a nonzero scalar,
    \item addition to one row of $A$ a scalar multiple of another (different) row, and
    \item interchange of two rows of $A$.
\end{enumerate}
%
If $e$ is an elementary row operation, we write $e(A)$ for the matrix obtained when applying $e$ to $A$. Clearly each elementary row operation $e$ has an \enquote{inverse}, i.e. an elementary row operation $e'$ such that $e'(e(A)) = e(e'(A)) = A$. Two matrices $A,B \in \mat{m,n}{\fieldF}$ are called \keyword{row-equivalent} if $A$ is obtained by applying a finite sequence of elementary row operations to $B$ (and vice versa, though this need not be assumed since each elementary row operation has an inverse).

Clearly, if $A, B \in \mat{m,n}{\fieldF}$ are row-equivalent, then the systems of equations $Ax = 0$ and $Bx = 0$ are combination equivalent, hence have the same solutions.

An \keyword{elementary matrix} is a matrix obtained by applying a single elementary row operation to the identity matrix $I$. It is easy to show that if $e$ is an elementary row operation and $E = e(I) \in \mat{m}{\fieldF}$, then $e(A) = EA$ for $A \in \mat{m,n}{\fieldF}$. If also $B \in \mat{m,n}{\fieldF}$, then $A$ and $B$ are row-equivalent if and only if $A = PB$, where $P \in \mat{m}{\fieldF}$ is a product of elementary matrices.

We now show that every matrix is row-equivalent to a matrix with a particularly simple form: \cref{def:mydef} \cref{enumdef:myenum} \itemref{enumdef:myenum}

\begin{definition}\label{def:mydef}
    A matrix $H \in \mat{m,n}{\fieldF}$ is called \keyword{row-reduced} if
    %
    \begin{enumdefinition}
        \item \label{enumdef:myenum} the first nonzero entry of each nonzero row in $H$ is $1$, and
        \item each column of $H$ containing the leading nonzero entry of some row has all its other entries equal $0$.
    \end{enumdefinition}
    %
    If $H$ is row-reduced, it is called a \keyword{row-reduced echelon matrix} if it also has the following properties:
    %
    \begin{enumdefinition}[resume]
        \item Every row of $H$ only containing zeroes occur below every row which has a nonzero entry, and
        \item if rows $1, \ldots, r$ are the nonzero rows of $H$, and if the leading nonzero entry of row $i$ occurs in column $k_i$, then $k_1 < \cdots < k_r$.
    \end{enumdefinition}
\end{definition}

\begin{proposition}[My proposition]
    Every matrix in $\mat{m,n}{\fieldF}$ is row-equivalent to a unique row-reduced echelon matrix.
\end{proposition}

\begin{proof}
    This is my proof!
\end{proof}

\begin{lemma}
    Every matrix in $\mat{m,n}{\fieldF}$ is row-equivalent to a unique row-reduced echelon matrix.
\end{lemma}

\begin{corollary}
    Every matrix in $\mat{m,n}{\fieldF}$ is row-equivalent to a unique row-reduced echelon matrix.
\end{corollary}

\begin{example}
    Every matrix in $\mat{m,n}{\fieldF}$ is row-equivalent to a unique row-reduced echelon matrix.
\end{example}

\begin{remark}
    Every matrix in $\mat{m,n}{\fieldF}$ is row-equivalent to a unique row-reduced echelon matrix.
\end{remark}
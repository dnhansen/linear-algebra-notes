\chapter{Preface}

These notes cover aspects of linear algebra that I have not found satisfactory expositions of elsewhere. We generally restrict ourselves to the finite-dimensional case, unless results can be generalised without significant effort. For instance, in the context of inner product spaces there is of course no loss in generality by restricting to the real or the complex numbers, and the elementary theory of Hilbert space adjoints is not simplified substantially by the assumption of finite dimension, so we make no such assumption. On the other hand, we only prove the spectral theorem for normal operators on finite-dimensional spaces.

Throughout we let $\fieldF$ denote an arbitrary field, $\fieldK$ a field that is either the real or the complex numbers, and $R$ a commutative ring. Unless otherwise specified, vector spaces will be vector spaces over $\fieldF$, and modules will be left modules over $R$. Furthermore, sesquilinear forms are linear in their \emph{second} entry. This is rarely relevant, but it seems more natural both in the theory of duality of spaces equipped with sesquilinear forms (see {par:Riesz-map}) and in the representation of sesquilinear forms with matrices (see {par:sesquilinear-form-matrix-representation}).
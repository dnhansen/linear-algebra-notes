\chapter{Bases and coordinates}

\section{Bases}

If $\calV$ is a subset of $V$, the \keyword{span} of $\calV$, denoted $\Span\calV$ or $\gen{\calV}$, is the smallest subspace of $V$ containing $\calV$. Equivalently, it is the set of all linear combinations
%
\begin{equation*}
    \alpha_1 v_1 + \cdots + \alpha_n v_n,
\end{equation*}
%
where $\alpha_i \in \fieldF$ and $v_i \in \calV$. We say that $\calV$ is \keyword{linearly independent} if any linear relation
%
\begin{equation*}
    \alpha_1 v_1 + \cdots + \alpha_n v_n = 0
\end{equation*}
%
among elements $v_i$ in $\calV$ can only be satisfied if $\alpha_1 = \cdots = \alpha_n = 0$. If $\calV$ is both linearly independent and a spanning set, then we call it a \keyword{basis} for $V$. We have the following characterisation of bases:

\begin{propositionnoproof}
    A subset $\calV \subseteq V$ is a basis for $V$ if and only if
    %
    \begin{equation*}
        V
            = \bigdirsum_{v \in \calV} \gen{v}.
    \end{equation*}
\end{propositionnoproof}

We next prove that bases always exist, but first we need a different characterisation of bases. An element $v \in V$ is an \keyword{essentially unique} linear combination of the elements in $\calV$ if there is an up to ordering unique way to express $v$ as a linear combination of elements in $\calV$. It is easy to see that $\calV$ is linearly independent if and only if every nonzero $v \in \gen{\calV}$ is an essentially unique linear combination of the elements in $\calV$.

\begin{proposition}
    Let $\calV$ be a subset of $V$. The following are equivalent:
    %
    \begin{enumproposition}
        \item \label{enum:linearly-independent-and-span} $\calV$ is linearly independent and spans $V$.
        \item \label{enum:essentially-unique-linear-combination} Every nonzero $v \in V$ is an essentially unique linear combination of vectors in $\calV$.
        \item \label{enum:minimal-spanning-set} $\calV$ is a minimal spanning set.
        \item \label{enum:maximal-linearly-independent-set} $\calV$ is a maximal linearly independent set.
    \end{enumproposition}
\end{proposition}

\begin{proof}
\begin{proofsec*}
    \item[\itemref{enum:linearly-independent-and-span} $\iff$ \itemref{enum:essentially-unique-linear-combination}]
    This follows easily as mentioned above.

    \item[\itemref{enum:linearly-independent-and-span} $\iff$ \itemref{enum:minimal-spanning-set}]
    If \itemref{enum:linearly-independent-and-span} holds and a proper subset $\calV'$ of $\calV$ spanned $V$, then any element of $\calV \setminus \calV'$ is a linear combination of elements in $\calV'$, so $\calV$ is not linearly independent. Conversely, if $\calV$ is a minimal spanning set but is not linearly independent, then some $v \in \calV$ is a linear combination of the other elements in $\calV$, so $\calV \setminus \{v\}$ is also a spanning set.

    \item[\itemref{enum:linearly-independent-and-span} $\iff$ \itemref{enum:maximal-linearly-independent-set}]
    Again assuming \itemref{enum:linearly-independent-and-span}, if $\calV$ were not maximal there would be some $v \in V \setminus \calV$ such that $\calV \union \{v\}$ were linearly independent. But then $v$ would not be a linear combination of elements in $\calV$. Conversely, if $\calV$ is a maximal linearly independent set that did not span $V$, then there would be some $v \in V \setminus \calV$ that is not a linear combination of elements in $\calV$. But then $\calV \union \{v\}$ is also linearly independent.
\end{proofsec*}
\end{proof}


\begin{theorem}
    \label{prop:basis-existence}
    Let $V$ be a vector space. If $\calI \subseteq V$ is linearly independent, $\calS \subseteq V$ is a spanning set, and $\calI \subseteq \calS$, then there is a basis $\calV$ for $V$ with $\calI \subseteq \calV \subseteq \calS$.
\end{theorem}

\begin{proof}
    Let $\calA$ be the collection of linearly independent subsets $\calJ$ of $V$ with $\calI \subseteq \calJ \subseteq \calS$. If $\calC$ is a chain in $\calA$, then
    %
    \begin{equation*}
        \calU
            = \bigunion \calC
    \end{equation*}
    %
    is linearly independent and satisfies $\calI \subseteq \calU \subseteq \calS$, so it lies in $\calA$. Hence every chain in $\calA$ has an upper bound, so it has a maximal element $\calV$. This is linearly independent since it lies in $\calA$, and it is also a spanning set by maximality, hence it is a basis.
\end{proof}


\begin{corollary}
    Every vector space has a basis.
\end{corollary}

\begin{proof}
    Let $\calI = \emptyset$ and $\calS = V$ in \cref{prop:basis-existence}.
\end{proof}


We next turn to the concept of the \keyword{dimension} of a vector space. Our presentation will focus on finite-dimensional vector spaces.

\begin{proposition}
    If the vectors $v_1, \ldots, v_n$ in $V$ are linearly independent, and the vectors $w_1, \ldots, w_m$ span $V$, then $n \leq m$.
\end{proposition}

\begin{proof}
    List the vectors as follows:
    %
    \begin{equation*}
        w_1, \ldots, w_m; v_1, \ldots, v_n.
    \end{equation*}
    %
    We transform this list such that the collection of vectors on the left-hand side of the semicolon always span $V$, and such that the vectors on the right-hand side are always linearly independent. Note that $v_1$ is a linear combination of the $w_j$, implying that we may add $v_1$ to the left-hand side and remove one of the $w_j$ (which, by reindexing, we may assume is $w_1$) and still have a spanning set. We simultaneously remove $v_1$ from the right-hand side. That is, we obtain
    %
    \begin{equation*}
        v_1, w_2, \ldots, w_m; v_2, \ldots, v_n.
    \end{equation*}
    %
    If $m < n$, then applying this process recursively will eventually exhaust the $w_j$, at which point we would have
    %
    \begin{equation*}
        v_1, \ldots, v_m; v_{m+1}, \ldots, v_n.
    \end{equation*}
    %
    But this is not possible, since $v_n$ does not lie in the span of $v_1, \ldots, v_m$. Hence $n \leq m$.
\end{proof}


\begin{corollarynoproof}
    If $V$ has a finite spanning set, then all bases for $V$ have the same cardinality.
\end{corollarynoproof}
%
This in fact holds for arbitrary vector spaces, though the proof is significantly more involved (cf. \cite[Theorem~1.12]{romanlinalg}).

Since bases always exist and all bases have the same cardinality, the following definition makes sense:

\begin{definition}[Dimension]
    The \keyword{dimension} of a vector space $V$, written $\dim V$, is the cardinality of any basis for $V$.
\end{definition}


We now turn to a different characterisation of the dimension of finite-dimensional vector spaces. Below we write $\dim V = \infty$ if the dimension of the vector space $V$ is infinite. A \keyword{series} of subspaces $U_i$ of $V$ is a finite or infinite decreasing sequence
%
\begin{equation*}
    V
        = U_0
        \supsetneq U_1
        \supsetneq U_2
        \supsetneq \cdots.
\end{equation*}
%
If the sequence is finite, then the \keyword{length} of the series is the number of strict inclusions. If the sequence is infinite, then we say that the length of the series is $\infty$. The maximal length of a series of subspaces of $V$ is denoted $l(V)$.

In the proposition below, we write $\dim V = \infty$ if the dimension of $V$ is infinite.

\begin{proposition}
    Let $V$ be a vector space. Then $\dim V = l(V)$.
\end{proposition}

\begin{proof}
    First assume that $V$ is finite-dimensional, and let $\calV = (v_1, \ldots, v_n)$ be a basis for $V$. Then there is a series
    %
    \begin{equation*}
        V
            = \gen{v_1, \ldots, v_n}
            \supsetneq \gen{v_1, \ldots, v_{n-1}}
            \supsetneq \cdots
            \supsetneq \gen{v_1}
            \supsetneq 0
    \end{equation*}
    %
    of subspaces of $V$, so $\dim V \leq l(V)$. Conversely, let
    %
    \begin{equation*}
        V
            = U_0
            \supsetneq U_1
            \supsetneq U_2
            \supsetneq \cdots
    \end{equation*}
    %
    be a series of subspaces of $V$. If the series ends with $0$, remove it. Hence all subspaces in the series are nontrivial. Then choose for each $i$ an element $v_i \in U_i \setminus U_{i+1}$, and collect them in a set $\calI$. It is clear that $\calI$ is linearly independent, hence finite. Thus the series is also finite with length $\card{\calI}-1$. Adding back $0$ to the series we obtain a series that is at least as long as the original sequence, and that is of length $\card{\calI} \leq \dim V$. Since the sequence was arbitrary, $l(V) \leq \dim V$.

    Next assume that $V$ is infinite-dimensional. Then $V$ contains a sequence $(v_i)_{i\in\naturals}$ that is linearly independent, so the series
    %
    \begin{equation*}
        V
            \supseteq \genset{v_i}{i \in \naturals}
            \supsetneq \genset{v_i}{i \geq 2}
            \supsetneq \genset{v_i}{i \geq 3}
            \supsetneq \cdots
    \end{equation*}
    %
    is infinite, and $l(V) = \infty$. Conversely, assume that $V$ has an infinite series. As above we construct a linearly independent set $\calI$ whose size equals the length of the sequence. Thus $V$ contains an infinite linearly independent set, so $\dim V = \infty$.
\end{proof}


\section{Coordinate maps and matrices}

Every matrix $A \in \mat{m,n}{\fieldF}$ gives rise to a map $M_A \colon \fieldF^n \to \fieldF^m$ given by $M_A v = Av$. The next result shows that every linear map $\fieldF^n \to \fieldF^m$ arises in this way:

\begin{proposition}
    \label{prop:smr-properties}
    Let $(e_1, \ldots, e_n)$ be the standard basis for $\fieldF^n$. The map
    %
    \begin{align*}
        \calM \colon \lin(\fieldF^n, \fieldF^m) &\to \mat{m,n}{\fieldF}, \\
        T &\mapsto \bigl( Te_1 \mid \cdots \mid Te_n \bigr),
    \end{align*}
    %
    is a linear isomorphism with inverse $A \mapsto M_A$. The matrix $\smr{T}$ is called the \keyword{standard matrix representation} of $T$. If $T \colon \fieldF^n \to \fieldF^m$ and $S \colon \fieldF^m \to \fieldF^l$ are linear maps, then
    %
    \begin{enumproposition}
        \item \label{enum:smr-vector-multiplication} $Tv = \smr{T}v$ for all $v \in \fieldF^n$.
        
        \item \label{enum:smr-of-identity-map} $\smr{\id_{\fieldF^n}} = I$.

        \item \label{enum:smr-multiplicative} $\smr{S \circ T} = \smr{S} \smr{T}$.

        \item \label{enum:smr-invertibility} $T$ is invertible if and only if $\smr{T}$ is invertible, in which case $\smr{\inv{T}} = \inv{\smr{T}}$.
    \end{enumproposition}
\end{proposition}

\begin{proof}
    The map $A \mapsto M_A$ is clearly linear, so to prove the first point it suffices to show that this is the inverse of $\calM$. Let $T \in \lin(\fieldF^n,\fieldF^m)$. Then
    %
    \begin{equation*}
        M_{\smr{T}} \colvec{\alpha_1 \\ \vdots \\ \alpha_n}
            = \smr{T} \colvec{\alpha_1 \\ \vdots \\ \alpha_n}
            = \bigl( Te_1 \mid \cdots \mid Te_n \bigr) \colvec{\alpha_1 \\ \vdots \\ \alpha_n}
            = \sum_{i=1}^n \alpha_i Te_i
            = T \colvec{\alpha_1 \\ \vdots \\ \alpha_n}
    \end{equation*}
    %
    for $\alpha_1, \ldots, \alpha_n \in \fieldF$. Conversely, for $A \in \mat{m,n}{\fieldF}$ we have
    %
    \begin{equation*}
        \smr{M_A}
            = \bigl( M_A e_1 \mid \cdots \mid M_A e_n \bigr)
            = \bigl( A e_1 \mid \cdots \mid A e_n \bigr)
            = A,
    \end{equation*}
    %
    since $Ae_i$ is the $i$th column of $A$. We prove the remaining claims:
    %
    \begin{proofsec}
        \item[\itemref{enum:smr-vector-multiplication}]
        Simply notice that $Tv = M_{\smr{T}}v = \smr{T}v$.

        \item[\itemref{enum:smr-of-identity-map}]
        This is obvious from the definition of $\calM$.

        \item[\itemref{enum:smr-multiplicative}]
        Let $v \in \fieldF^n$ and notice that
        %
        \begin{equation*}
            \smr{S \circ T}v
                = (S \circ T) v
                = S(Tv)
                = S(\smr{T}v)
                = \smr{S}\smr{T}v
        \end{equation*}
        %
        by \itemref{enum:smr-vector-multiplication}. Since this holds for all $v$, the claim follows.

        \item[\itemref{enum:smr-invertibility}]
        This follows easily from \itemref{enum:smr-of-identity-map} and \itemref{enum:smr-multiplicative}.
    \end{proofsec}
\end{proof}

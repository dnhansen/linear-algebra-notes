\chapter{Complexification}

If $W$ is a complex vector space, then we may restrict the scalar multiplication $\complex \prod W \to W$ to a map $\reals \prod W \to W$. When we equip $W$ with this restricted scalar multiplication instead of the original one, we call the resulting space the \keyword{real version} of $W$ and denote it by $W_\reals$.

Conversely, if $V$ is a real vector space then we define the \keyword{complexification} of $V$ as the vector space $V^\complex$ whose underlying set is $V \prod V$, and which is equipped with componentwise addition and the complex scalar multiplication
%
\begin{equation*}
    (\alpha + \iu \beta)(v,u)
        = (\alpha v - \beta u, \alpha u + \beta v),
\end{equation*}
%
for $\alpha,\beta \in \reals$ and $v,u \in V$. Notice that the map $v \mapsto (v,0)$ is injective (and real linear), that $(v,0) + (w,0) = (v+w,0)$, and that $\alpha(v,0) = (\alpha v,0)$ for $\alpha \in \reals$, so $V^\complex$ contains an isomorphic copy of $V$, and we may identify elements $v \in V$ with elements $(v,0) \in V^\complex$. Furthermore, notice that $(v,u) = (v,0) + \iu (u,0)$, so by the above identification we may write $(v,u) = v + \iu u$.


We briefly study the relationship between a real vector space and its complexification.

\begin{proposition}
    If $\calB$ is a basis for $V$, then $\calB^\complex = \set{b + \iu 0}{b \in \calB}$ is a basis for $V^\complex$. In particular, $\dim_\reals V = \dim_\complex V^\complex$.
\end{proposition}

\begin{proof}
    Let $v + \iu u \in V^\complex$. Then there are real numbers $\alpha_b$ and $\beta_b$ (finitely many nonzero) such that $v = \sum_{b \in \calB} \alpha_b b$ and $u = \sum_{b \in \calB} \beta_b b$. But then
    %
    \begin{equation*}
        v + \iu u
            = \sum_{b \in \calB} \alpha_b b
                + \iu \sum_{b \in \calB} \beta_b b
            = \sum_{b \in \calB} (\alpha_b + \iu \beta_b) b
            = \sum_{b \in \calB} (\alpha_b + \iu \beta_b) (b + \iu 0),
    \end{equation*}
    %
    so $\calB^\complex$ spans $V^\complex$. Furthermore, if $v + \iu u = 0$, then the previous computation shows that $\sum_{b \in \calB} \alpha_b b = 0 = \sum_{b \in \calB} \beta_b b$. Linear independence of $\calB$ then implies that $\alpha_b = \beta_b = 0$ for all $b \in \calB$.
\end{proof}

\begin{example}
    Notice that $(\reals^n)^\complex \cong \complex^n$. The above proposition then implies that the standard basis for $\reals^n$ gives rise to a basis for $\complex^n$, and we notice that this is precisely the standard basis.
\end{example}


We now show how to extend linear maps defined between real vector spaces to the complexifications of those spaces. If $T \colon V \to W$ is a linear map between real vector spaces, then we define the complexification of $T$ by
%
\begin{align*}
    T^\complex \colon V^\complex &\to W^\complex, \\
    v + \iu u &\mapsto Tv + \iu Tu.
\end{align*}
%
That is, $T^\complex$ is just the product map $T \prod T$. This is easily seen to be complex-linear.

\begin{proposition}
    \label{prop:complexification-eigenvalue}
    Let $V$ be a real vector space, and let $T \in \lin(V)$. If $\lambda \in \reals$ is an eigenvalue of the complexification $T^\complex$ of $T$, then $\lambda$ is also an eigenvalue of $T$. Furthermore, if $v + \iu u \in E_{T^\complex}(\lambda)$ then $v,u \in E_T(\lambda)$.
\end{proposition}
%
Note that this does not mean that $v$ and $u$ are eigenvectors of $T$ since they might be zero. But if $v + \iu u$ is an eigenvector of $T^\complex$, then at least one of $v$ and $u$ is nonzero and hence an eigenvector of $T$.

\begin{proof}
    Let $v + \iu u \in V^\complex$ be an eigenvector of $T^\complex$ corresponding to $\lambda$. Then
    %
    \begin{equation*}
        Tv + \iu Tu
            = T^\complex (v + \iu u)
            = \lambda(v + \iu u)
            = \lambda v + \iu \lambda u.
    \end{equation*}
    %
    It follows that $Tv = \lambda v$ and $Tu = \lambda u$ as desired.
\end{proof}


If $V$ is finite-dimensional and $\calV$ is an ordered basis for $V$, then $\calV^\complex$ carries the obvious ordering. Since $V$ and $V^\complex$ have the same dimension, the following result is not surprising:

\begin{proposition}
    Let $V$ and $W$ be a finite-dimensional real vector spaces, and consider $T \colon V \to W$. If $\calV = (v_1, \ldots, v_n)$ and $\calW$ are ordered bases of $V$ and $W$ respectively, then
    %
    \begin{equation*}
        \mr{\calW^\complex}{T^\complex}{\calV^\complex}
            = \mr{\calW}{T}{\calV}.
    \end{equation*}
\end{proposition}

\begin{proof}
    By \cref{enum:mr-explicit-formula}, the $i$th column of $\mr{\calW^\complex}{T^\complex}{\calV^\complex}$ is given by
    %
    \begin{equation*}
        \coordvec{T^\complex (v_i + \iu 0)}{\calW^\complex}
            = \coordvec{Tv_i + \iu 0}{\calW^\complex}
            = \coordvec{Tv_i}{\calW},
    \end{equation*}
    %
    that is, the $i$th column of $\mr{\calW}{T}{\calV}$, which proves the claim.
\end{proof}


Finally, if $V$ is a real normed space, then we define a norm on $V^\complex$ by the equation
%
\begin{equation}
    \label{eq:complexification-norm}
    \norm{v + \iu u}^2
        = \norm{v}^2 + \norm{u}^2.
\end{equation}
%
Furthermore, if $V$ is an inner product space, then we define an inner product on $V^\complex$ by
%
\begin{equation}
    \label{eq:complexification-inner-product}
    \inner{v + \iu u}{x + \iu y}
        = \inner{v}{x}
          + \inner{u}{y}
          + \iu (\inner{v}{y} - \inner{u}{x}).
\end{equation}
%
The norm induced by this inner product agrees with the norm defined by \cref{eq:complexification-norm}. Notice that the identity \cref{eq:complexification-inner-product} holds in any \emph{complex} inner product space, where the notation $v + \iu u$ instead means the sum of $v$ and the scalar product of $\iu$ and $u$ (recalling that our sesquilinear forms are linear in the \emph{second} entry).
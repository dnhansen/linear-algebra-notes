% Document setup
\documentclass[article, a4paper, 11pt, oneside]{memoir}
\usepackage[utf8]{inputenc}
\usepackage[T1]{fontenc}
\usepackage[UKenglish]{babel}

% Document info
\newcommand\doctitle{Roman, \emph{Advanced Linear Algebra}}
\newcommand\docauthor{Danny Nygård Hansen}

% Formatting and layout
\usepackage[autostyle]{csquotes}
\renewcommand{\mktextelp}{(\textellipsis\unkern)}
\usepackage[final]{microtype}
\usepackage{xcolor}
\frenchspacing
\usepackage{latex-sty/articlepagestyle}
\usepackage{latex-sty/articlesectionstyle}

% Fonts
\usepackage{amssymb}
\usepackage[largesmallcaps,partialup]{kpfonts}
\DeclareSymbolFontAlphabet{\mathrm}{operators} % https://tex.stackexchange.com/questions/40874/kpfonts-siunitx-and-math-alphabets
\linespread{1.06}
% \let\mathfrak\undefined
% \usepackage{eufrak}
\DeclareMathAlphabet\mathfrak{U}{euf}{m}{n}
\SetMathAlphabet\mathfrak{bold}{U}{euf}{b}{n}
% https://tex.stackexchange.com/questions/13815/kpfonts-with-eufrak
\usepackage{inconsolata}

% Hyperlinks
\usepackage{hyperref}
\definecolor{linkcolor}{HTML}{4f4fa3}
\hypersetup{%
	pdftitle=\doctitle,
	pdfauthor=\docauthor,
	colorlinks,
	linkcolor=linkcolor,
	citecolor=linkcolor,
	urlcolor=linkcolor,
	bookmarksnumbered=true
}

% Equation numbering
\numberwithin{equation}{chapter}

% Footnotes
\footmarkstyle{\textsuperscript{#1}\hspace{0.25em}}

% Mathematics
\usepackage{latex-sty/basicmathcommands}
\usepackage{latex-sty/framedtheorems}
\usepackage{tikz-cd}
\tikzcdset{arrow style=math font} % https://tex.stackexchange.com/questions/300352/equalities-look-broken-with-tikz-cd-and-math-font
\usetikzlibrary{babel}

% Lists
\usepackage{enumitem}
\setenumerate[0]{label=\normalfont(\alph*)}

% Bibliography
\usepackage[backend=biber, style=authoryear, maxcitenames=2, useprefix]{biblatex}
\addbibresource{references.bib}

% Title
\title{\doctitle}
\author{\docauthor}

\newcommand{\calM}{\mathcal{M}}
\newcommand{\calB}{\mathcal{B}}
\DeclareMathOperator{\sign}{sgn}
\DeclareMathOperator{\adj}{adj}


\usepackage{listofitems}
\setsepchar{,}

\makeatletter

\newcommand{\mat@dims}[1]{%
    \readlist*\@dims{#1}%
    \ifnum \@dimslen=1
        \def\@dimsout{\@dims[1]}%
    \else
        \def\@dimsout{\@dims[1], \@dims[2]}%
    \fi
    \@dimsout
}

\newcommand{\matgroup}[3]{\mathrm{#1}_{#2}(#3)}
\newcommand{\GL}[2]{\matgroup{GL}{#1}{#2}}
\newcommand{\trans}{^{\top}}
\newcommand{\mat}[2]{\calM_{\mat@dims{#1}}(#2)}

\makeatother

%% Framed exercise environment

\mdfdefinestyle{swannexercise}{%
    skipabove=0.5em plus 0.4em minus 0.2em,
	skipbelow=0.5em plus 0.4em minus 0.2em,
	leftmargin=-5pt,
	rightmargin=-5pt,
	innerleftmargin=5pt,
	innerrightmargin=5pt,
	innertopmargin=5pt,
	innerbottommargin=4pt,
	linewidth=0pt,
	splittopskip=1.2em minus 0.2em,
	splitbottomskip=0.5em plus 0.2em minus 0.1em,
	backgroundcolor=backgroundcolor,
	frametitlebackgroundcolor=titlecolor,
	frametitlefont={\scshape},
    theoremseparator={},
    % theoremspace={},
	frametitleaboveskip=3pt,
	frametitlebelowskip=2pt
}

\mdtheorem[style=swannexercise]{exerciseframed}{Exercise}

\let\oldexerciseframed\exerciseframed
\renewcommand{\exerciseframed}{%
  \crefalias{theorem}{exerciseframed}%
  \oldexerciseframed}

\usepackage{listofitems}

% \settocdepth{subsection}
\renewenvironment{exerciseframed}[1][]{%
    \setsepchar{.}%
    \readlist*\mylist{#1}%
    \def\exlabel{\mylist[1].\mylist[2]}%
    % \refstepcounter{exerciseframed}%
    % \addcontentsline{toc}{subsection}{Exercise \smalllabel}%
    \begin{exerciseframed*}[\exlabel]%
    \label{ex:#1}%
}{%
    \end{exerciseframed*}%
}

% https://tex.stackexchange.com/a/23491/63353
\newcommand{\RNum}[1]{\uppercase\expandafter{\romannumeral #1\relax}}

\newcommand{\exref}[1]{%
    \setsepchar{.}%
    \readlist*\mylist{#1}%
    \ifnum \arabic{chapter}=\mylist[1]
        \def\mylabel{\mylist[2].\mylist[3]}%
    \else
        \def\mylabel{\RNum{\mylist[1]}.\mylist[2].\mylist[3]}%
    \fi
    \hyperref[ex:#1]{Exercise~\mylabel}%
}

\theoremstyle{nonumberplain}
\theoremsymbol{\ensuremath{\square}}
\newtheorem{solution}{Solution}

\let\oldsolution\solution
\renewcommand{\solution}{%
  \crefalias{theorem}{solution}%
  \oldsolution}

\newcommand{\solutionlabelfont}[1]{{\normalfont\color{linkcolor}#1}}
\newlist{solutionsec}{enumerate}{1}
\setlist[solutionsec]{leftmargin=0pt, parsep=0pt, listparindent=\parindent, font=\solutionlabelfont, label=(\alph*), labelsep=0pt, labelwidth=20pt, itemindent=20pt, align=left, itemsep=10pt}


\begin{document}

\maketitle

\chapter{Vector Spaces}

\begin{exerciseframed}[1.11]
    Show that if $S$ is a subspace of a vector space $V$, then $\dim S \leq \dim V$. Furthermore, if $\dim S = \dim V < \infty$ then $S = V$.
\end{exerciseframed}

\begin{solution}
    Let $\calB$ be a basis for $S$. Then this is linearly independent as a subset of $V$, hence is contained in a basis $\calB'$ for $V$ by Theorem~1.9. Then $\calB \subseteq \calB'$, so it follows that $\dim S \leq \dim V$.

    Now assume that $\dim S = \dim V < \infty$. Then $\card{\calB} = \card{\calB'}$, but since each basis is finite and one is contained in the other, we must have $\calB = \calB'$. Hence $S = V$.
\end{solution}


\begin{exerciseframed}[1.12]
    Suppose that $V = U \oplus S_1 = U \oplus S_2$. What can you say about the relationship between $S_1$ and $S_2$? What can you say if $S_1 \subseteq S_2$?
\end{exerciseframed}

\begin{solution}
    By Theorem~3.6, all complements of $U$ are isomorphic, so we always have $S_1 \cong S_2$. Assume that $S_1 \subseteq S_2$, and let $s \in S_2$. Then $s = u + s'$ for some $u \in U$ and $s' \in S_1$. But then $s'$ also lies in $S_2$, so since the sum $U \oplus S_2$ is direct we have $u = 0$.
\end{solution}


\chapter{Linear Transformations}

\newcommand{\im}{\operatorname{im}}
\newcommand{\calL}{\mathcal{L}}

\begin{exerciseframed}[2.15]
    Suppose that $T \in \calL(V,W)$.
    %
    \begin{enumerate}
        \item Given $L \in \calL(U,W)$, show that there exists an $R \in \calL(V,U)$ with $T = LR$ if and only if $\im T \subseteq \im L$:
        %
        \begin{equation*}
            \begin{tikzcd}
                V
                    \ar[rr, "T", bend left]
                    \ar[r, "R", dashed, swap]
                & U
                    \ar[r, "L", swap]
                & W
            \end{tikzcd}
        \end{equation*}

        \item Given $R \in \calL(V,U)$, show that there exists an $L \in \calL(U,W)$ with $T = LR$ if and only if $\ker R \subseteq \ker T$:
        %
        \begin{equation*}
            \begin{tikzcd}
                V
                    \ar[rr, "T", bend left]
                    \ar[r, "R", swap]
                & U
                    \ar[r, "L", dashed, swap]
                & W
            \end{tikzcd}
        \end{equation*}
    \end{enumerate}
    %
    In particular, both monomorphisms and epimorphisms split.
\end{exerciseframed}

\begin{solution}
\begin{solutionsec}
    \item Write $W = \ker L \oplus M$ for some subspace $M \subseteq W$. Then the restriction $L|_M \colon M \to \im L$ is bijective, so let $R = (L|_M)\inv T$, which is well-defined since $\im T \subseteq \im L$.
    
    \item Write $V = \ker R \oplus M$ for some subspace $M \subseteq V$. Then $R|_M \colon M \to \im R$ is bijective. Writing $U = \im R \oplus N$ for some subspace $N \subseteq U$, let $L = T \circ [(R|_M)\inv, 0]$. For $v \in \ker R \subseteq \ker T$ we have
    %
    \begin{equation*}
        LRv
            = L(0)
            = 0
            = Tv,
    \end{equation*}
    %
    and for $v \in M$ we have
    %
    \begin{equation*}
        LRv
            = T(R|_M)\inv Rv
            = Tv,
    \end{equation*}
    as required.
\end{solutionsec}
\end{solution}

\newcommand{\field}{\mathbb{F}}

\begin{exerciseframed}[2.22]
    Let $T \in \calL(V)$. If $TS = ST$ for all $S \in \calL(V)$, show that $T = \alpha \id_V$ for some $\alpha \in \field$. I.e., the centre of the ring $\calL(V)$, with multiplication given by function composition, is the subspace $\langle \id_V \rangle$.
\end{exerciseframed}

\begin{solution}
    This is obvious if $\dim V \in \{0,1\}$, so assume that $\dim V \geq 2$.

    First let $v \in V \setminus \{0\}$ and write $V = \langle v \rangle \oplus U$ for some subspace $U$, and define $S \in \calL(V)$ by letting $Sv = v$ and $Su = 0$ for $u \in U$. If $T$ and $S$ commute, then
    %
    \begin{equation*}
        STv
            = TSv
            = Tv.
    \end{equation*}
    %
    Hence $Tv \in \langle v \rangle$ (which includes the possibility that $Tv = 0$).

    Next assume that $v,w \in V$ are linearly independent, write $V = \langle v, w \rangle \oplus U_2$ for some subspace $U_2$, and define $S$ by letting $Sv = w$, $Sw = v$ and $Su = 0$. Let $\alpha \in \field$ be such that $Tv = \alpha v$. Then
    %
    \begin{equation*}
        Tw
            = TSv
            = STv
            = \alpha Sv
            = \alpha w.
    \end{equation*}
    %
    Hence $Tv = \alpha v$ for all $v \in V$ as desired.
\end{solution}


\chapter{The Isomorphism Theorems}

\begin{exerciseframed}[3.18]
    Let $S$ be a subspace of $V$. Prove that $(V/S)^* \cong S^0$.
\end{exerciseframed}

\begin{solution}
    Let $\pi \colon V \to V/S$ be the quotient map, such that for every $\phi \in (V/S)^*$ we have
    %
    \begin{equation*}
        \begin{tikzcd}[column sep=tiny]
            V
                \ar[rr, "\phi \circ \pi"]
                \ar[rd, "\pi", swap]
            && \field \\
            & V/S
                \ar[ru, "\phi", swap]
        \end{tikzcd}
    \end{equation*}
    %
    Consider the map $(V/S)^* \to V^*$ given by $\phi \mapsto \phi \circ \pi$. This is injective by the universal property of quotients. Also by this property, a functional $\psi \in V^*$ factors through $\pi$ if and only if $S \subseteq \ker \psi$, i.e. if $\psi \in S^0$. Hence the image of the above map is precisely $S^0$.
\end{solution}


\addtocounter{chapter}{4}
\chapter{Eigenvalues and Eigenvectors}

\newcommand{\calV}{\mathcal{V}}
\newcommand{\mr}[3]{{}_{#1}[#2]_{#3}}
\newcommand{\spec}{\operatorname{Spec}}

\begin{exerciseframed}[8.21]
    A pair of linear operators $T,S \in \calL(V)$ (with $\dim V < \infty$) is \emph{simultaneously diagonalisable} if there is an ordered basis $\calV$ for $V$ such that $\mr{\calV}{T}{\calV}$ and $\mr{\calV}{S}{\calV}$ are both diagonal. Prove that two diagonalisable operators $T$ and $S$ are simultaneously diagonalisable if and only if they commute.
\end{exerciseframed}

\newenvironment{displaytheorem}{%
	\begin{displayquote}\itshape%
}{%
	\end{displayquote}%
}

\begin{solution}
    First assume that $T$ and $S$ are simultaneously diagonalisable, and write $\calV = (v_1, \ldots, v_n)$. Then each $v_i$ is an eigenvector for both $T$ and $S$, so let $Tv_i = \lambda_i v_i$ and $Sv_i = \mu_i v_i$. Then
    %
    \begin{equation*}
        TSv_i
            = \mu_i Tv_i
            = \mu_i \lambda_i v_i
            = \lambda_i \mu_i v_i
            = \lambda_i Sv_i
            = STv_i,
    \end{equation*}
    %
    so $T$ and $S$ commute. Alternatively, we may simply notice that the matrix representations of $T$ and $S$ commute (since they are diagonal), so
    %
    \begin{equation*}
        \mr{\calV}{TS}{\calV}
            = \mr{\calV}{T}{\calV} \cdot \mr{\calV}{S}{\calV}
            = \mr{\calV}{S}{\calV} \cdot \mr{\calV}{T}{\calV}
            = \mr{\calV}{ST}{\calV},
    \end{equation*}
    %
    and hence $TS = ST$.

    In order to prove the converse we will need a couple of lemmas, starting with:
    %
    \begin{displaytheorem}
        Assume that the subspace $U$ is invariant under $T \in \calL(V)$. If $v_1, \ldots, v_k \in V$ are eigenvectors of $T$ corresponding to distinct eigenvalues and $v_1 + \cdots + v_k \in U$, then all $v_i$ lie in $U$.
    \end{displaytheorem}
    %
    We prove this claim by induction. For $k = 1$ this is obvious, so assume that it holds for $k-1$. Then let $\lambda_i$ be the eigenvalue corresponding to $v_i$, and put $u = v_1 + \cdots + v_k$. Notice that
    %
    \begin{equation*}
        Tu - \lambda_1 u
            = (\lambda_2 - \lambda_1)v_2 + \cdots + (\lambda_k - \lambda_1)v_k.
    \end{equation*}
    %
    The left-hand side lies in $U$. Hence each summand on the right-hand side lies in $U$ by induction, and since the eigenvalues are distinct so do $v_2, \ldots, v_k$. Since $U$ is a subspace, $v_1$ does as well.
    
    \begin{displaytheorem}
        If $T \in \calL(V)$ is diagonalisable and the subspace $U$ is invariant under $T$, then $T|_U \in \calL(U)$ is also diagonalisable.
    \end{displaytheorem}
    %
    Each $u \in U$ is a finite sum of eigenvectors of $T$ corresponding to distinct eigenvalues. Hence each eigenvector also lies in $U$, so
    %
    \begin{equation*}
        U
            = \bigoplus_{\lambda \in \spec T} U \intersect E_T(\lambda)
            = \bigoplus_{\lambda \in \spec T|_U} E_{T|_U}(\lambda),
    \end{equation*}
    %
    where the last equality follows since $U \intersect E_T(\lambda)$ is precisely the set of eigenvectors of $T$ corresponding to $\lambda$ that lie in $U$, i.e. the eigenvectors of $T|_U$ corresponding to $\lambda$. 
    
    Finally assume that $TS = ST$. If $v \in E_T(\lambda)$, then
    %
    \begin{equation*}
        TSv
            = STv
            = \lambda Sv,
    \end{equation*}
    %
    so also $Sv \in E_T(\lambda)$. In other words, every eigenspace of $T$ is invariant under $S$. By the lemma above, $S$ restricted to $E_T(\lambda)$ is thus diagonalisable, hence has a basis $\calV_\lambda$ of eigenvectors of $S$. But these are also eigenvectors of $T$. Then $\calV = \bigunion_{\lambda \in \spec T} \calV_\lambda$ is a basis for $V$ consisting of simultaneous eigenvectors of $T$ and $S$.
\end{solution}

\end{document}